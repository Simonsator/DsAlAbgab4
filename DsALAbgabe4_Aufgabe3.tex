\section{Substitionsmethode Reloaded}
\subsection{}
zZ: Jede Teilmenge M von $\mathbb{T}$ hat eine kleinste obere und eine größte untere Schranke. \\
S,T $\in$ M $\subseteq $ $\mathbb{T}$ \\
S $\preceq$ T $\leftrightarrow$ $\forall$n: $S(n)<=T(n)$|<= ist eine Totalordnung auf $\mathbb{R}$ \\
$\Rightarrow$ $\forall$R,T: R $\preceq$ T $\lor$ T $\preceq$ R \\
$\Rightarrow$ $\exists$S $\succeq$ T $\in$ M : (T $\preceq$ S) $\land$ ($\forall$Y $\in$ M : T $\preceq$ y $\Rightarrow$ S $\preceq$ Y) \\
$\Rightarrow$ Jede Teilmene M von $\mathbb{T}$ hat eine kleinste obere Schranke. \\ \\S,T $\in$ M $\subseteq $ $\mathbb{T}$ \\
S $\preceq$ T $\leftrightarrow$ $\forall$n: $S(n)<=T(n)$|<= ist eine Totalordnung auf $\mathbb{R}$ \\
$\Rightarrow$ $\forall$R,T: R $\preceq$ T $\lor$ T $\preceq$ R \\
$\Rightarrow$ $\exists$S $\preceq$ T $\in$ M : (T $\succeq$ S) $\land$ ($\forall$Y $\in$ M : T $\succeq$ y $\Rightarrow$ S $\succeq$ Y) \\
$\Rightarrow$ Jede Teilmene M von $\mathbb{T}$ hat eine größte untere Schranke. \\ \\
Somit hat jede Teilmenge M von $\mathbb{T}$ eine kleinste untere- und eine größte obere Schranke im Sinne der Halbordnung $\preceq$. Somit ist ($\mathbb{T},\preceq$) ein vollständiger Verband.
\subsection{}
zZ: X $\preceq$ Y $\Rightarrow$ $\Psi$(X) $\preceq$ $\Psi$(Y) \\
$\Leftrightarrow$ X $\preceq$ Y $\Rightarrow$ 2X($\lfloor$n/2$\rfloor$) + n <= 2Y($\lfloor$n/2$\rfloor$) + n \\
Fall 1: n = 0 X $\preceq$ Y $\Rightarrow$ 0 <= 0 ist eine wahre Aussagen. \\
Fall 2: n $\in$ $\mathbb{N}$ X $\preceq$ Y $\Rightarrow$ 2X($\lfloor$n/2$\rfloor$)+ n <= 2Y($\lfloor$n/2$\rfloor$)+ n | n $\in$ $\mathbb{N}$ sei c:= ($\lfloor$n/2$\rfloor$ \\
$\Leftrightarrow$ X $\preceq$ Y $\Rightarrow$ 2X(c) + n <= 2Y(c) + n | -n \\
$\Leftrightarrow$ X $\preceq$ Y $\Rightarrow$ 2X(c)<= 2Y(c) | :2 \\
$\Leftrightarrow$ X $\preceq$ Y $\Rightarrow$ X(c)<= Y(c) ist eine warhe Aussage. Somit ist gezeigt, dass $\Psi$ monoton bezüglich $\preceq$ ist.
\subsection{}
\subsection{}
\subsection{}
\subsection{}
\subsection{}
Die Aufgabe e) fiel mir leichter, da die Methode in der Vorlesung vorgestellt wurde.